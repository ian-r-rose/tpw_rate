\documentclass[a4paper,12pt]{article}
\usepackage[utf8]{inputenc}

%opening
\title{Response to referee}
\author{Ian Rose and Bruce Buffett}
\date{}

\begin{document}

\maketitle

We thank Yanick Ricard for his thoughtful review.
In addition to several smaller suggestions and corrections,
his review centers on our treatment of the reference frames relevant to the scaling of true polar wander (TPW).
We broadly agree with his criticism, and have substantially reworked our discussion of reference frames to address it.
Several of the key equations have been corrected, and we have changed some of the notation for clarity,
but the overall conclusions of the manuscript are unchanged.

In particular, we have done the following to clarify and correct the treatment of reference frames:
\\
\\
1. We have added a section (Section 2.2) which makes the differences between reference frames explict.
We identify three frames which are important to the problem: 
(1) the inertial, nonrotating frame, 
(2) A body-fixed geographic frame, rotating with respect to the inertial frame with relative rotation vector $\mathbf{\Omega}$, and
(3) The frame described by the principal axes of the convective moment of inertia $\mathbf{E}$, which we call the $\mathbf{E}$-frame.
This frame rotates slowly with respect to the body-fixed frame with rotation vector $\mathbf{\Psi}$.
\\
\\
2. We have added a new figure (Figure 1) which illustrates the different reference frames.
\\
\\
3. We have changed the discussion in Section 2.5 to clarify that the rate of TPW is measured in the body-fixed frame.
The rate of TPW we denote by $\dot{\Theta}$. The physics describing the magnitude of $\dot{\Theta}$ are 
more naturally expressed in the $\mathbf{E}$-frame, with colatitude $\theta$ and longitude $\phi$. 
For much of the discussion we make the simplifying assumption that $\phi=0$.
If the $\mathbf{E}$-frame is not rotating with respect to the body fixed frame ($\mathbf{\Psi}=0$) then $\dot{\Theta} = \dot{\theta}$.
\\
\\
4. We have updated the discussion of the time evolution of the mismatch angle $\theta$ (Section 4.2)
in light of the changes to the reference frames. The characteristic size of $\theta$ is set by 
the competition of its decay via TPW and its growth via $\mathbf{\Psi}$.
This is reflected in Equations (43) and (44).
Our scaling furnishes estimates of the sizes of these two processes.
\\
\\
Besides the discussion of reference frames, we have addressed the following other issues raised in the review:

\begin{itemize}
\item We have added the clarifying intermediate step in Equation (7).
\item We have added Equation (21) on the source of the gravitational field.
\item We have changed the notation of the ratio of centrifugal to gravitational forces from
the fluid-dynamics-inspired ``Froude number'' to the symobl $m$ more commonly used in geodesy.
\item We have clarified discussion of transforming the linear momentum equation to the
angular momentum equation (Equations (25)-(29)). The review criticized this transformation as not very useful.
While we agree that these equations are not particularly easy to evaluate in practice,
they do serve two useful purposes for our scaling: First, they establish the overall consistency
between the approximations made for the linear momentum equation (Section 3) and those made for
the angular momentum equation (Section 2). Second, Equations (23)-(29) demonstrate the
origin of the nondimensionalization that we use for our scaling. In particular, Equations (28)-(29)
shows the origin of $\alpha \Delta T$ as an independent nondimensional number.
\item We have added additional note on the effect of dynamic compensation on the value of $(1+k_f^L)$
\item The review suggested that stirring and mixing can populate lengthscales smaller than $d$
in the convecting system, where $d$ is the injection lengthscale, roughly set by the thickness of
the boundary layers. While mixing can do this, we argue that those lengthscales are
quickly homogenized by thermal diffusion, whereas thermal anomalies at larger lengthscales 
persist for a much longer time.

\end{itemize}



\end{document}
