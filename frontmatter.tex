\chapter{Rates of true polar wander on convecting planets}
\label{chap:intro}


Mass redistribution in the convecting mantle of a planet can cause perturbations in its moment of inertia tensor. 
Conservation of angular momentum dictates that these perturbations can change the direction of the rotation vector of the planet, a process known as true polar wander (TPW). 
Although the existence of TPW on Earth is well-verified, its rate and magnitude over geologic time scales remain controversial. 
Here we present scaling analyses and numerical simulations of TPW due to mantle convection over a range of parameter space relevant to planetary interiors. 
For simple rotating convection, the most important parameters are the Rayleigh number, the rotation rate, and the size of relative density fluctuations (i.e. thermal expansivity times the temperature variations). 
We identify timescales for the growth of moment of inertia perturbations due to convection and for their relaxation due to true polar wander. 
These timescales, as well as the relative sizes of convective anomalies, control the rate and magnitude of TPW.
This analysis also clarifies the nature of so called ``inertial interchange'' TPW events, and when they are likely to occur.
Finally, we discuss implications for large-scale TPW in Earth's past, which has been suggested to be important for life and climate history.
