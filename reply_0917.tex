\documentclass[a4paper,12pt]{article}
\usepackage[utf8]{inputenc}

%opening
\title{Response to reviewer 1 (Victor Tsai)}
\author{}
\date{}

\begin{document}

\maketitle

We thank Victor for his thorough review of our paper.
We reproduce his comments here, with inline replies.

\begin{quote}
In abstract, ``Earths'' should be ``Earth's''
  
Line 44: ``must be also'' should be ``must also''
\end{quote}
These have been fixed.

\begin{quote}
Line 106-107: It is not clear what is meant by this statement. Do the authors mean ``no \_further\_ TPW is permitted'' after the fluid limit is achieved? Clearly there is finite TPW for the infinite time point relative to a given earlier time point.
\end{quote}
This clarification is a nice suggestion, and we have included it.

\begin{quote}
Line 132: ``calculate to'' should be ``calculate''
\end{quote}
This has been fixed.
     
\begin{quote}
Line 395-400: A result similar to this is discussed in Section 2, paragraph 15 of Tsai \& Stevenson (2007) and their Equation 4. In particular, they also find that many factors cancel out when comparing how TPW scales with Rayleigh number. The result in Equation 43 appears to be different than the T\&S result, but it would be useful to compare the present result to that earlier result, and note any similarities and differences.
\end{quote}
We have included a few sentences discussing the similar result in T\&S.
In particular, there is a different scaling for the dependence of the moment-of-inertia
anomalies upon the Rayleigh number.

\begin{quote}
Line 409-410: There appears to be a jump in topic, without a transition. A transition would be welcome.
\end{quote}
We have reorganized the sentences here, moving them to the start of the appropriate section rather than the end of the preceding section. We hope that this clarifies the narrative.

\begin{quote}
Line 410-415: Apparently there still appears to be differences in semantics between the authors’ usage and what I am familiar with as common usage. As far as I am aware, TPW is always defined as the relative reorientation of the rotation axis and the geographical axes. IITPW is thus a specific type of (hypothesized) TPW in which this relative reorientation reaches angles close to 90 degrees. It is true that the mechanism proposed behind “inertial interchange” relates to theta being close to 90 degrees, but I think the authors should still be clearer about this. Perhaps it is a difference between ``II'' and ``IITPW'': ``II'' is the mechanism (related to theta); ``IITPW'' is a resulting TPW (not related to theta directly, but indirectly through its suspected mechanism). Although this is indeed a semantic issue, it is an important semantic issue that I am sure many readers might be confused by. Partly, this paragraph is also confusing because many authors (like Tsai \& Stevenson) clearly recognize that theta can be large (close to 90 degrees), but still have a limit on TPW rates, which makes the earlier statement misleading.
\end{quote}
This semantic issue has been a bit sticky, but we take Victor's point.
We have clarified some of the language around use of IITPW, using ``inertial interchange''
when referring to the forcing for this particular mode of polar wander, and IITPW
for the 90 degree response to that forcing. In particular, we have changed some of the language
around lines 410-423 and 475-480.

\begin{quote}
Line 485-486: It is not clear why this statement is true. In particular, Equation 33 has dW/dt=0 if Lambda\_31 = 0. So, it is not clear why TPW is fast. Presumably, it is because Lambda\_31 grows fast enough that it quickly becomes large enough to drive significant TPW, but this is not obviously true. In any case, some further discussion would be warranted. It would also be useful if TPW rates were shown on Fig4, since it is difficult to tell how fast the TPW is (due to the compressed x axis).
\end{quote}
We have added an additional sentence clarifying our intent. We do indeed mean the $\Lambda_{31}$
grows fast enough that it quickly is able to drive TPW.
Furthermore, we have added the suggested panel to Figure 4, showing TPW rates of a few $^\circ$/Myr.
These rates are consistent with the rates derived in our scaling analyses.

\begin{quote}
Line 502-504: I believe the Tsai \& Stevenson (2007) suggestion is not so different from what is proposed. T\&S certainly emphasized that immediately after interchange, the forcing is zero, but that it can grow as long as there is sufficient time for that growth to occur. As mentioned in the previous review, the question appears to be how representative Fig4 is. In their response, the authors respond that it is ``very'' representative at high Rayleigh number. I do not doubt that, but the text is not very clear about this, and many readers might misinterpret the authors’ statements as saying Fig4 is representative of all TPW (i.e., even lower Rayleigh number TPW, despite the labeling of Rayleigh number in the figure caption). I would suggest that it behooves the authors to be crystal clear(er) about such statements.
\end{quote}
We have added an additional panel to Figure 4, and accompanying text in that figure caption, as well as at line 577 regarding the representativeness of this result.
          
\begin{quote}
Line 509: “the the” should be “the”
\end{quote}
This has been fixed.

\begin{quote}
Line 565: Equation 46 appears to be the same formula as in Tsai \& Stevenson (2007), but the numerical result appears to be 2-3 times larger (c.f. 2.4 degrees/Myr). Is this because of different choices of viscosities or moment of inertia magnitudes? Given the similarities/differences of the results, the Tsai \& Stevenson (2007) result should be referenced here and it would be useful to discuss the differences/similarities.
\end{quote}
This is indeed a similar formula to that of Tsai \& Stevenson (2007).
We were unable to reproduce the $2.4^\circ$/Myr figure, and so are unable to comment on specific differences, except to note that T\&S states
"The maximum TPW speed is given by equation (18) and is $2.4^\circ \textrm{Myr}^{-1}$ multiplied by a factor of order 1."
\\
\\
We appreciate Victor's thoughtful suggestions.

\end{document}
