% gjilguid2e.tex

%\documentclass[extra,onecolumn]{gji}
\documentclass[extra]{gji}
\usepackage{timet}
\usepackage{amsmath}

\title[TPW Rates]
  {Rates of true polar wander in convecting planets}
\author[I. Rose and B. Buffett]
  {Ian Rose$^1$ and Bruce A. Buffett$^1$ \\
  $^1$ Department of Earth \& Planetary Science, University of California, Berkeley, CA 94720, USA.  E-mail: ian.rose@berkeley.edu
  }
\date{}
\pagerange{\pageref{firstpage}--\pageref{lastpage}}
\volume{}
\pubyear{}

\let\leqslant=\leq

\begin{document}

\label{firstpage}

\maketitle

\begin{summary}
Abstract goes where?
\end{summary}

\begin{keywords}
Earth rotation variations; Mantle processes; Dynamics: convection currents and mantle plumes; Planetary interiors; Numerical solutions; Paleomagnetism applied to tectonics.
\end{keywords}

\section{Introduction}


\section{Governing equations}

As mantle convection and rotational dynamics of planetary bodies are usually considered separately, some extra care must be taken to establish self-consistent governing equations.  
Here we consider a planet in a rotating reference frame with no internal heating in the incompressible Boussinesq approximation.  The equations for mass, momentum, and energy then read

\begin{equation}
\nabla \cdot {\bf v} = 0
\label{conserve_mass}
\end{equation}

\begin{equation}
\begin{aligned}
 \rho_0 \left[ \frac{D \bf v}{D t} + \frac{d \bf \Omega}{dt} \times {\bf r} +  2 {\bf \Omega} \times {\bf v} + {\bf \Omega \times \Omega \times r} \right] \\ = - \nabla P + \nabla \cdot \left( \eta \nabla_s {\bf v} \right) + \rho {\bf g}
\label{navier_stokes}
\end{aligned}
\end{equation}

\begin{equation}
\frac{\partial T}{\partial t} + {\bf v} \cdot \nabla T = \kappa \nabla^2 T
\label{energy}
\end{equation}

along with the equation of state

\begin{equation}
\rho = \rho_0 \left( 1 - \alpha T \right)
\label{eos}
\end{equation}

Dimensional analysis of this system (c.f. \citet{barenblatt1996scaling}) requires six nondimensional numbers to characterize it.

\begin{table}
\caption{Parameters for rotating mantle convection}
\label{parameters}
\begin{tabular}{@{}lcc}
Symbol & Definition\\
\hline
$R_i$ & inner radius \\
$R_o$ & outer radius \\
$\Omega_0$ & reference rotation rate \\
$\eta$ & viscosity \\
$\kappa$ & thermal diffusivity \\
$\alpha$ & thermal expansivity \\
$g_0$ & reference gravity \\
$\rho_0$ & reference density \\
$\Delta T$ & temperature difference \\ 
\end{tabular}
\end{table}

\begin{table}
\caption{Nondimensional numbers with approximate Earthlike values}
\label{nondim}
\begin{tabular}{@{}lcccc}
Symbol &  Number & Definition & Approximate value \\
\hline
Ra & Rayleigh &  $\rho_0 g_0 \alpha \Delta T R_o^3/\eta \kappa$ & $10^7$\\
Pr & Prandtl & $\eta/\rho_0 \kappa$ & $10^{23}$ \\
Ek & Ekman & $\eta/\rho_0 \Omega_0 R_{o}^2$ & $10^8$ \\
Fr & Froude & $\Omega_0^2 R_o/g_0$ & $10^{-3}$ \\
$\Gamma$ & Density deficit & $\alpha \Delta T$ & $10^{-2}$ \\
A & Aspect ratio & $R_i/R_o$ & $0.55$ \\
\end{tabular}
\end{table}
 



\section{Scaling}

\section{Rates of true polar wander}

\section{Numerical results}

\section{Discussion}


\begin{acknowledgments}
\end{acknowledgments}

\bibliographystyle{gji}
\bibliography{tpw_rate.bib}


\label{lastpage}

\end{document}
