% gjilguid2e.tex

%\documentclass[extra,onecolumn]{gji}
\documentclass[extra]{gji}
\usepackage{timet}
\usepackage{amsmath}

\title[TPW Rates]
  {Rates of true polar wander in convecting planets}
\author[I. Rose and B. Buffett]
  {Ian Rose$^1$ and Bruce A. Buffett$^1$ \\
  $^1$ Department of Earth \& Planetary Science, University of California, Berkeley, CA 94720, USA.  E-mail: ian.rose@berkeley.edu
  }
\date{}
\pagerange{\pageref{firstpage}--\pageref{lastpage}}
\volume{}
\pubyear{}

\let\leqslant=\leq

%in case I want to add detail to any derivation
\newif\ifdetail
\detailfalse
%\detailtrue

\begin{document}

\label{firstpage}

\maketitle

\begin{summary}
Mass redistribution in the convecting mantle of a planet can cause perturbations in its moment of inertia tensor. 
Conservation of angular momentum dictates that these perturbations can change the direction of the rotation vector of the planet, a process known as true polar wander (TPW). 
Although the existence of TPW on Earth is well-verified, its rate and magnitude over geologic time scales remain controversial. 
Here we present scaling analyses and numerical simulations of TPW due to mantle convection over a range of parameter space relevant to planetary interiors. 
For simple rotating convection, the most important parameters are the Rayleigh number, the rotation rate, and the size of relative density fluctuations (i.e. thermal expansivity times the temperature variations). 
We identify timescales for the growth of moment of inertia perturbations due to convection and for their relaxation due to true polar wander. 
These timescales, as well as the relative sizes of convective anomalies, control the rate and magnitude of TPW.
This analysis also clarifies the nature of so called ``inertial interchange'' TPW events, and when they are likely to occur.
Finally, we discuss implications for large-scale TPW in Earth's past, which has been suggested to be important for life and climate history.
\end{summary}

\begin{keywords}
Earth rotation variations; Mantle processes; Dynamics: convection currents and mantle plumes; Planetary interiors; Numerical solutions; Paleomagnetism applied to tectonics.
\end{keywords}

\section{Introduction}


\section{Governing equations}

As mantle convection and rotational dynamics of planetary bodies are usually considered separately, some extra care must be taken to establish self-consistent governing equations.  
Here we consider an isoviscous planet in a rotating reference frame with no internal heating in the incompressible Boussinesq approximation.  The equations for mass, momentum, and energy then read

\begin{equation}
\nabla \cdot {\bf u} = 0
\label{conserve_mass}
\end{equation}

\begin{equation}
\begin{aligned}
 \rho{\bf \Omega \times \Omega \times r}= - \nabla P + \nabla \cdot \left( \eta \varepsilon ({\bf u}) \right) + \rho {\bf g}
\label{navier_stokes}
\end{aligned}
\end{equation}

\begin{equation}
\frac{\partial T}{\partial t} + {\bf u} \cdot \nabla T = \kappa \nabla^2 T
\label{energy}
\end{equation}

 where the parameters are defined in Table~\ref{parameters} and $\varepsilon({\bf v}) = 1/2 \; ( \partial v_i / \partial x_j + \partial v_j / \partial x_i )$ corresponds to the symmetrized velocity gradient.
 In addition we use the simple equation of state

\begin{equation}
\rho = \rho_0 \left( 1 - \alpha T \right)
\label{eos}
\end{equation}

Note that here we retain the centrifugal term, which is normally either neglected or absorbed into a modified pressure. 
Dimensional analysis of this system (cf. \citet{barenblatt1996scaling}) requires five nondimensional numbers to characterize it.

\ifdetail
\subsection{Buckingham Pi theorem and nondimensionalization}
The Buckingham Pi theorem (essentially an application of the rank-nullspace theorem from linear algebra) allows us to determine the number of nondimensional numbers for this problem.  
We count the number of fundamental units for the problem, in this case mass (kg), length (m), time (s), and temperature (K), as well as the number of parameters in use for this problem.  
These parameters are listed in Table \ref{parameters}, and subtracting the ...

\begin{table}
\caption{Nondimensionalizations for variables, primed variables denote dimensional variables}
\label{nondim_convert}
\begin{tabular}{@{}lll}
$x^\prime$ &=& $R \;\; x$ \\
$t^\prime$ &=& $R^2/\kappa \;\; t$ \\
$T^\prime$ &=& $\Delta T \;\; T$ \\
$\rho^\prime$ &=& $\rho_0 \;\; \rho$\\
$v^\prime$ &=& $\kappa/R \;\; v$ \\
$P^\prime$ &=& $\eta \kappa/R^2 \;\; P$ \\
$\Omega^\prime$ &=& $\Omega_0 \;\; \Omega$ \\
$g^\prime$ &=& $g_0 \;\; g$
\end{tabular}
\end{table}

If we plug these nondimensionalizations into the governing equations we find that the incompressibility constraint remains unchanged.  The temperature equation and equation of state become

\begin{equation}
\frac{\partial T}{\partial t} + {\bf v} \cdot \nabla T = \nabla^2 T
\end{equation}

\begin{equation}
\rho = 1 - \alpha \Delta T \;\;  T
\end{equation}

We may furthermore define a hydrostatic reference state where the density is the reference density everywhere and the velocity is zero, i.e.

\begin{equation}
 \rho_0{\bf \Omega \times \Omega \times r}= - \nabla P_0 + \rho_0 {\bf g}
\end{equation}

Subtracting this from the momentum equation and defining a dynamic pressure $P^* = P - P_0$, we find

\begin{equation}
\begin{aligned}
 - & \Omega_0^2  \alpha  \Delta T T R \; {\bf \Omega \times \Omega \times r} = - \eta \kappa / R^2 \; \nabla P^* \\ 
&+ \eta \kappa / R^2 \; \nabla \cdot \left( \eta \varepsilon ({\bf v}) \right) - \alpha \Delta T g_0 \; {\bf g}
\end{aligned}
\end{equation}

\begin{equation}
\begin{aligned}
 - & \mathrm{Ra \; Fr}\; T \; {\bf \Omega \times \Omega \times r} = - \nabla P^* \\ 
&+ \; \nabla \cdot \left( \eta \varepsilon ({\bf v}) \right) - \mathrm{Ra} \; T \; {\bf g}
\end{aligned}
\end{equation}

  
\fi


Convenient choices for these numbers are shown in Table \ref{nondim} along with approximate Earthlike values.
In addition to the more usual Rayleigh and Prandtl numbers we include a Froude number, characterizing the ratio of centrifugal to gravitational forces, and $\Gamma$, a nondimensional characterization of density variations.
Nondimensionalizing with these parameters we obtain
\begin{equation}
\begin{aligned}
 - \nabla P^* + \; \nabla^2{\bf v} - \mathrm{Ra} \; T \; {\bf g} + \mathrm{Ra \; Fr}\; T \;{\bf \Omega \times \Omega \times r} = 0
\end{aligned}
\end{equation}

where we have introduced a dynamic pressure $P^* = P - P_0$.

\begin{table}
\caption{Parameters for rotating mantle convection}
\label{parameters}
\begin{tabular}{@{}lcc}
Symbol & Definition\\
\hline
$R_i$ & inner radius \\
$R$ & outer radius \\
$\Omega_0$ & reference rotation rate \\
$\eta$ & viscosity \\
$\kappa$ & thermal diffusivity \\
$\alpha$ & thermal expansivity \\
$g_0$ & reference gravity \\
$\rho_0$ & reference density \\
$\Delta T$ & temperature difference \\ 
\end{tabular}
\end{table}

\begin{table}
\caption{Nondimensional numbers with approximate Earthlike values}
\label{nondim}
\begin{tabular}{@{}lcccc}
Symbol &  Number & Definition & Approximate value \\
\hline
Ra & Rayleigh &  $\rho_0 g_0 \alpha \Delta T R^3/\eta \kappa$ & $10^7$\\
Pr & Prandtl & $\eta/\rho_0 \kappa$ & $10^{23}$ \\
Fr & Froude & $\Omega_0^2 R/g_0$ & $10^{-3}$ \\
$\Gamma$ & Density deficit & $\alpha \Delta T$ & $10^{-2}$ \\
A & Aspect ratio & $R_i/R$ & $0.55$ \\
\end{tabular}
\end{table}
 



\section{Rotational dynamics}

We may write the nonlinear Liouville equations 
\begin{equation}
{\bf \Omega} \times {\bf \dot{\Omega} } = \frac{1}{\varepsilon I_0 \tau} {\bf \Omega} \times \left( {\bf E \cdot \Omega} \right)
\end{equation}

introducing a unit vector $\mitbf{\omega} = {\bf \Omega}/\|{\bf \Omega} \|$ we may solve this equation for ${\bf \dot{\mitbf{\omega}} }$:
\begin{equation}
 \dot{\mitbf{\omega}}  = \frac{1}{\varepsilon I_0 \tau} \left[ {\bf E \cdot \mitbf{\omega}} - \left( {\bf \mitbf{\omega} \cdot E \cdot \mitbf{\omega} } \right) \mitbf{\omega} \right]
\end{equation}

Note that the quantity in brackets is identical in form to the shear stress on a plane in classical elastostatics.
If we enter the coordinate system of the convective moment of inertia $\bf E$ with principal moments $\lambda_1 \le \lambda_2 \le \lambda_3$ and define the orientation of $\mitbf{\omega}$ with colatitude $\theta$ and longitude $\phi$ we can write this in a more illuminating form (after some tedious algebra):

\ifdetail

And here is the tedious algebra.  Let the prefactors $\frac{1}{\varepsilon I_0 \tau}$ be denoted by $A$, and the pole of the coordinate system ($\theta = 0$) associated with the eigenvector for $\lambda_3$.  Therefore we find that $\mitbf{\omega} = \left[ \sin{\theta} \cos{\phi} \;\; \sin{\theta} \sin{\phi} \;\; \cos{\theta} \right]^T$.  Plugging this in, we find a system of three equations, which we want to solve for $\dot{\theta}$ and $\dot{\phi}$:

\begin{equation}
\begin{aligned}
 & \cos{\theta}\cos{\phi} \dot{\theta}  - \sin{\theta}\sin{\phi} \dot{\phi} = \\
  &A \sin{\theta}\cos{\phi}\left[ \lambda_1 - \lambda_1 \sin^2{\theta}\cos^2{\phi} - \lambda_2 \sin^2{\theta}\sin^2{\phi} - \lambda_3 \cos^2{\theta} \right] \\
 &\cos{\theta}\sin{\phi} \dot{\theta}  + \sin{\theta}\cos{\phi} \dot{\phi} = \\
  &A \sin{\theta}\sin{\phi} \left[ \lambda_2 - \lambda_1 \sin^2{\theta}\cos^2{\phi} - \lambda_2 \sin^2{\theta}\sin^2{\phi} - \lambda_3 \cos^2{\theta} \right] \\
 - &\sin{\theta} \dot{\theta} = \\
  &A \cos{\theta} \left[ \lambda_3 - \lambda_1 \sin^2{\theta}\cos^2{\phi} - \lambda_2 \sin^2{\theta}\sin^2{\phi} - \lambda_3 \cos^2{\theta} \right] \\
\end{aligned}
\end{equation}

Now multiply the top equation by $\cos{\phi}$ and the second equation by $\sin{\phi}$ and add them together to get

\begin{equation}
\begin{aligned}
 & \cos{\theta}\cos^2{\phi} \dot{\theta}  - \sin{\theta}\sin{\phi} \cos{\phi} \dot{\phi} + \cos{\theta}\sin^2{\phi} \dot{\theta}  + \sin{\theta}\cos{\phi} \sin{\phi} \dot{\phi} = \\
  &A \sin{\theta}\cos^2{\phi}\left[ \lambda_1 - \lambda_1 \sin^2{\theta}\cos^2{\phi} - \lambda_2 \sin^2{\theta}\sin^2{\phi} - \lambda_3 \cos^2{\theta} \right] + \\
  &A \sin{\theta}\sin^2{\phi} \left[ \lambda_2 - \lambda_1 \sin^2{\theta}\cos^2{\phi} - \lambda_2 \sin^2{\theta}\sin^2{\phi} - \lambda_3 \cos^2{\theta} \right] \\
 & \cos{\theta}\dot{\theta} =  
   A \sin{\theta} \left[ \lambda_1 \cos^2{\phi} +  \lambda_2 \sin^2{\phi} \right] + \\
  &A \sin{\theta} \left[ - \lambda_1 \sin^2{\theta}\cos^2{\phi} - \lambda_2 \sin^2{\theta}\sin^2{\phi} - \lambda_3 \cos^2{\theta} \right] \\
\end{aligned}
\end{equation}

Now multiply the above by $\cos{\theta}$ and the third equation by $-\sin{\theta}$ and add them to get

\begin{equation}
\begin{aligned}
 & \cos^2{\theta}\dot{\theta} + \sin^2{\theta} \dot{\theta} =  
   A \sin{\theta} \cos{\theta} \left[ \lambda_1 \cos^2{\phi} +  \lambda_2 \sin^2{\phi} \right] + \\
  &A \sin{\theta} \cos{\theta} \left[ - \lambda_1 \sin^2{\theta}\cos^2{\phi} - \lambda_2 \sin^2{\theta}\sin^2{\phi} - \lambda_3 \cos^2{\theta} \right] - \\
  &A \cos{\theta} \sin{\theta} \left[ \lambda_3 - \lambda_1 \sin^2{\theta}\cos^2{\phi} - \lambda_2 \sin^2{\theta}\sin^2{\phi} - \lambda_3 \cos^2{\theta} \right] \\
\end{aligned}
\end{equation}

After a lot of cancellation, we find

\begin{equation}
\begin{aligned}
 \dot{\theta} &= A \sin{\theta} \cos{\theta} \left[ - \lambda_3 + \lambda_1 \cos^2{\phi} + \lambda_2 \sin^2{\phi} \right] \\
              &= - A \sin{\theta} \cos{\theta} \left[ (\lambda_3 - \lambda_1) \cos^2{\phi} + (\lambda_3 - \lambda_2) \sin^2{\phi} \right] \\
              &= -\frac{A}{2} \sin{2 \theta} \left[ (\lambda_3 - \lambda_1) \cos^2{\phi} + (\lambda_3 - \lambda_2) \sin^2{\phi} \right] \\
\end{aligned}
\end{equation}

Likewise we may multiply the first equation by $-\sin{\phi}$ and the second equation by $\cos{\phi}$ and add them to get

\begin{equation}
\begin{aligned}
 & \sin{\theta} \dot{\phi} = A \sin{\theta}\cos{\phi}\sin{\phi} \left( \lambda_2 - \lambda_1 \right ) \\
& \dot{\phi} = \frac{A}{2} \sin{2 \phi} (\lambda_2-\lambda_1)
\end{aligned}
\end{equation}

\else
\fi


\begin{equation}
\begin{aligned}
\dot{\theta} &= - \frac{1}{2 \varepsilon I_0 \tau} \sin{2 \theta} \left[ (\lambda_3-\lambda_1) \cos^2{\phi} + (\lambda_3-\lambda_2) \sin^2{\phi} \right] \\
\dot{\phi} &= \;\; \frac{1}{2 \varepsilon I_0 \tau} \sin{2 \phi} \; (\lambda_2 - \lambda_1)
\end{aligned}
\label{milankovitch}
\end{equation}

Without loss of generality we may choose the coordinate system such that $\phi=0$:
\begin{equation}
\dot{\theta} = - \frac{1}{2 \varepsilon \tau}\sin{2 \theta} \frac{(\lambda_3-\lambda_1)}{I_0}
\label{simple_milankovitch}
\end{equation}

These equations are a version of what has been called the ``Milankovitch theorem'' \citep{munk1960rotation}.  
Written this way, it is clear that the important quantities are $\theta$ and the differences between the eigenvalues of the convective moment $\bf E$.
It is useful to define a nondimensional eigenvalue difference (or ``eigengap'') $\Lambda_{ij} = (\lambda_i - \lambda_j)/I_0$.  
With this we may rewrite this in terms of the nondimensional numbers defined above, where 
$\varepsilon \sim \mathrm{Fr}$ and $\tau \sim \eta / \rho g R = \frac{R^2}{\kappa} \Gamma/\mathrm{Ra}$. 

\begin{equation}
\frac{\kappa}{R^2} \; \dot{\theta} \sim - \frac{\mathrm{Ra}}{\Gamma \mathrm{Fr}} \Lambda_{31} \sin{2 \theta}
\end{equation}

At this point we do not have estimates for the characteristic magnitudes of $\Lambda_{ij}$ or $\theta$, both of which are crucial for estimating characteristic rates of TPW.
They represent, respectively, the size of convective anomalies in the moment of inertia tensor and the angular mismatch between where the 
convective moment

\section{Scaling}


The nonhydrostatic moment of inertia is defined
\begin{equation}
J_{ij} = \int_{V_S} \rho \left( r_q r_q \delta_{ij} - r_i r_j \right) 
\end{equation}

where $V_S$ is the reference spherical volume of the mantle.
If we substitute in Equation \ref{eos} we find

\begin{equation}
J_{ij} = I_0 \delta_{ij} + \int_{V_S} \rho_0 \alpha T \left( r_q r_q \delta_{ij} - r_i r_j \right) = I_0 \delta_{ij} + E_{ij} 
\end{equation}

Nondimensionalizing the integral for $E_{ij}$ we find

\begin{equation}
J_{ij} = I_0 \delta_{ij} + \int_{V_S^\prime} \alpha T^\prime \left( r_q^\prime r_q^\prime \delta_{ij} - r_i^\prime r_j^\prime \right) 
\end{equation}

\section{Numerical results}

\section{Discussion}




\begin{acknowledgments}
\end{acknowledgments}

\bibliographystyle{gji}
\bibliography{tpw_rate.bib}


\appendix

\section{Degree-two moments}

We may explicitly draw a connection between the moment of inertia of a rotating object and it's degree-two density structure.  The moment of inertia tensor may be written in index notation
\begin{equation}
I_{ij} = \int_V \rho \left( r_q r_q \delta_{ij} - r_i r_j \right) d^3 {\bf r}
\label{inertia}
\end{equation}

where $\bf r$ is the Eulerian coordinate, $\rho$ is the density, and $V$ is the volume of the material.  The degree-two density structure may be expressed as a quadrupole-moment tensor (cf. \citet{jackson1998classical}):

\begin{equation}
Q_{ij} = \int_V \rho \left( 3 r_i r_j - r_q r_q \delta_{ij} \right) d^3 {\bf r}
\end{equation}

The commutator of these tensors, $\bf QI - IQ$, is zero, indicating that they may be simultaneously diagonalized.  

\ifdetail
We may demonstrate that the commutator is zero by using the definition of the commutator and expanding:
\begin{equation} 
\begin{aligned}
{\bf IQ-QI} = &I_{ik} Q_{kj} - Q_{ik} I_{kj} \\
 = &\int_V \rho \left( r_q r_q \delta_{ik} - r_i r_k \right) d^3 {\bf r} \int_V \rho \left( 3 r_k r_j - r_q r_q \delta_{kj} \right) d^3 {\bf r} \\
 &- \int_V \rho \left( r_q r_q \delta_{kj} - r_k r_j \right) d^3 {\bf r} \int_V \rho \left( 3 r_i r_k - r_q r_q \delta_{ik} \right) d^3 {\bf r} \\
 = &3 \delta_{ik} \int_V \rho r_q r_q d^3 {\bf r} \int_V \rho r_k r_j d^3 {\bf r} \\
 &-\delta_{kj} \delta_{ik} \int_V \rho r_q r_q d^3 {\bf r} \int_V \rho r_q r_q d^3 {\bf r} \\
 &-3 \int_V \rho r_i r_k d^3 {\bf r} \int_V \rho r_k r_j d^3 {\bf r} \\
 &+\delta_{kj} \int_V \rho r_i r_k d^3 {\bf r} \int_V \rho r_q r_q d^3 {\bf r} \\
 - &3 \delta_{kj} \int_V \rho r_q r_q d^3 {\bf r} \int_V \rho r_i r_k d^3 {\bf r} \\
 &+\delta_{ik} \delta_{kj} \int_V \rho r_q r_q d^3 {\bf r} \int_V \rho r_q r_q d^3 {\bf r} \\
 &+3 \int_V \rho r_k r_j d^3 {\bf r} \int_V \rho r_i r_k d^3 {\bf r} \\
 &-\delta_{ik} \int_V \rho r_k r_j d^3 {\bf r} \int_V \rho r_q r_q d^3 {\bf r} \\
 = \;&3 \int_V \rho r_q r_q d^3 {\bf r} \int_V \rho r_i r_j d^3 {\bf r} \\
 &+ \int_V \rho r_i r_j d^3 {\bf r} \int_V \rho r_q r_q d^3 {\bf r} \\
 - &3 \int_V \rho r_q r_q d^3 {\bf r} \int_V \rho r_i r_j d^3 {\bf r} \\
 &- \int_V \rho r_i r_j d^3 {\bf r} \int_V \rho r_q r_q d^3 {\bf r} \\
 & = 0
\end{aligned}
\end{equation}
\else
\fi

We may therefore go into the principle coordinate system of both tensors to find
\begin{equation}
{\bf I} = \begin{bmatrix}
\int_V \rho (y^2+z^z)\\
\int_V \rho (x^2+z^2) \\
\int_V \rho (x^2+y^2) 
\end{bmatrix}^T {\bf 1}
\end{equation}

\begin{equation}
{\bf Q} = \begin{bmatrix}
\int_V \rho (2 x^2 - y^2 - z^z)\\
\int_V \rho (2 y^2 - x^2 -z^2)\\
\int_V \rho (2 z^2 - x^2 - y^2) 
\end{bmatrix}^T {\bf 1}
\end{equation}

where $\bf 1$ is the identity matrix.  If we let the diagonal elements of $\bf I$ be $\lambda_1, \lambda_2, \lambda_3$, we may rewrite $\bf Q$ as

\begin{equation}
{\bf Q} = \begin{bmatrix}
(\lambda_2-\lambda_1) + (\lambda_3-\lambda_1)\\
(\lambda_1-\lambda_2) + (\lambda_3-\lambda_2)\\
(\lambda_1-\lambda_3) + (\lambda_2-\lambda_3)\\
\end{bmatrix}^T {\bf 1}
\label{maccullagh}
\end{equation}

which shows the connection between the degree-two component of the density field and the principle moments of inertia.  This may be seen as a form of MacCullagh's formula \citep{stacey1977physics}.

\label{lastpage}

\end{document}
