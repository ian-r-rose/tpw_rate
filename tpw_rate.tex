% gjilguid2e.tex

%\documentclass[extra,onecolumn]{gji}
\documentclass[extra]{gji}
\usepackage{timet}
\usepackage{amsmath}

\title[TPW Rates]
  {Rates of true polar wander in convecting planets}
\author[I. Rose and B. Buffett]
  {Ian Rose$^1$ and Bruce A. Buffett$^1$ \\
  $^1$ Department of Earth \& Planetary Science, University of California, Berkeley, CA 94720, USA.  E-mail: ian.rose@berkeley.edu
  }
\date{}
\pagerange{\pageref{firstpage}--\pageref{lastpage}}
\volume{}
\pubyear{}

\let\leqslant=\leq

%in case I want to add detail to any derivation
\newif\ifdetail
\detailfalse

\begin{document}

\label{firstpage}

\maketitle

\begin{summary}
Abstract goes where?
\end{summary}

\begin{keywords}
Earth rotation variations; Mantle processes; Dynamics: convection currents and mantle plumes; Planetary interiors; Numerical solutions; Paleomagnetism applied to tectonics.
\end{keywords}

\section{Introduction}


\section{Governing equations}

As mantle convection and rotational dynamics of planetary bodies are usually considered separately, some extra care must be taken to establish self-consistent governing equations.  
Here we consider an isoviscous planet in a rotating reference frame with no internal heating in the incompressible Boussinesq approximation.  The equations for mass, momentum, and energy then read

\begin{equation}
\nabla \cdot {\bf v} = 0
\label{conserve_mass}
\end{equation}

\begin{equation}
\begin{aligned}
 \rho{\bf \Omega \times \Omega \times r}= - \nabla P + \nabla \cdot \left( \eta \varepsilon ({\bf v}) \right) + \rho {\bf g}
\label{navier_stokes}
\end{aligned}
\end{equation}

\begin{equation}
\frac{\partial T}{\partial t} + {\bf v} \cdot \nabla T = \kappa \nabla^2 T
\label{energy}
\end{equation}

 where the parameters are defined in Table~\ref{parameters} and $\varepsilon({\bf v}) = 1/2 \; ( \partial v_i / \partial x_j + \partial v_j / \partial x_i )$ corresponds to the symmetrized velocity gradient.
 In addition we use the simple equation of state

\begin{equation}
\rho = \rho_0 \left( 1 - \alpha T \right)
\label{eos}
\end{equation}

Note that here we retain the centrifugal term, which is normally either neglected or absorbed into a modified pressure. 
Dimensional analysis of this system (cf. \citet{barenblatt1996scaling}) requires five nondimensional numbers to characterize it.
Convenient choices for these numbers are shown in Table \ref{nondim} along with approximate Earthlike values.
In addition to the more usual Rayleigh and Prandtl numbers we include a Froude number, characterizing the ratio of centrifugal to gravitational forces, and $\Gamma$, a nondimensional characterization of density variations.

\begin{table}
\caption{Parameters for rotating mantle convection}
\label{parameters}
\begin{tabular}{@{}lcc}
Symbol & Definition\\
\hline
$R_i$ & inner radius \\
$R$ & outer radius \\
$\Omega_0$ & reference rotation rate \\
$\eta$ & viscosity \\
$\kappa$ & thermal diffusivity \\
$\alpha$ & thermal expansivity \\
$g_0$ & reference gravity \\
$\rho_0$ & reference density \\
$\Delta T$ & temperature difference \\ 
\end{tabular}
\end{table}

\begin{table}
\caption{Nondimensional numbers with approximate Earthlike values}
\label{nondim}
\begin{tabular}{@{}lcccc}
Symbol &  Number & Definition & Approximate value \\
\hline
Ra & Rayleigh &  $\rho_0 g_0 \alpha \Delta T R^3/\eta \kappa$ & $10^7$\\
Pr & Prandtl & $\eta/\rho_0 \kappa$ & $10^{23}$ \\
Fr & Froude & $\Omega_0^2 R/g_0$ & $10^{-3}$ \\
$\Gamma$ & Density deficit & $\alpha \Delta T$ & $10^{-2}$ \\
A & Aspect ratio & $R_i/R$ & $0.55$ \\
\end{tabular}
\end{table}
 



\section{Rates of true polar wander}

We may write the nonlinear Liouville equations 
\begin{equation}
{\bf \Omega} \times {\bf \dot{\Omega} } = \frac{1}{\varepsilon I_0 \tau} {\bf \Omega} \times \left( {\bf E \cdot \Omega} \right)
\end{equation}

introducing a unit vector $\mitbf{\omega} = {\bf \Omega}/\|{\bf \Omega} \|$ we may solve this equation for ${\bf \dot{\mitbf{\omega}} }$:
\begin{equation}
 \dot{\mitbf{\omega}}  = \frac{1}{\varepsilon I_0 \tau} \left[ {\bf E \cdot \mitbf{\omega}} - \left( {\bf \mitbf{\omega} \cdot E \cdot \mitbf{\omega} } \right) \mitbf{\omega} \right]
\end{equation}

Note that the quantity in brackets is identical in form to the shear stress on a plane in classical elastostatics.
If we enter the coordinate system of the convective moment of inertia $\bf E$ with principal moments $\lambda_1 \le \lambda_2 \le \lambda_3$ and define the orientation of $\mitbf{\omega}$ with colatitude $\theta$ and longitude $\phi$ we can write this in a more illuminating form (after some tedious algebra):
\begin{equation}
\begin{aligned}
\dot{\theta} &= - \frac{1}{2 \varepsilon I_0 \tau} \sin{2 \theta} \left[ (\lambda_3-\lambda_1) \cos^2{\phi} + (\lambda_3-\lambda_2) \sin^2{\phi} \right] \\
\dot{\phi} &= \;\; \frac{1}{2 \varepsilon I_0 \tau} \sin{2 \phi} \; (\lambda_2 - \lambda_1)
\end{aligned}
\label{milankovitch}
\end{equation}


These equations are a version of what has been called the ``Milankovitch theorem'' \citep{munk1960rotation}.  
Written this way, it is clear that the important quantities are $\theta$, $\phi$, and the differences of the eigenvalues of the convective moment $\bf E$.
Thus it remains to derive estimates for these.

\section{Scaling}

The nonhydrostatic moment of inertia is defined
\begin{equation}
J_{ij} = \int_{V_S} \rho \left( r_q r_q \delta_{ij} - r_i r_j \right) 
\end{equation}

where $V_S$ is the reference spherical volume of the mantle.
If we substitute in Equation \ref{eos} we find

\begin{equation}
J_{ij} = I_0 \delta_{ij} + \int_{V_S} \rho_0 \alpha T \left( r_q r_q \delta_{ij} - r_i r_j \right) = I_0 \delta_{ij} + E_{ij} 
\end{equation}

Nondimensionalizing the integral for $E_{ij}$ we find

\begin{equation}
J_{ij} = I_0 \delta_{ij} + \int_{V_S^\prime} \alpha T^\prime \left( r_q^\prime r_q^\prime \delta_{ij} - r_i^\prime r_j^\prime \right) 
\end{equation}

\section{Numerical results}

\section{Discussion}


\begin{acknowledgments}
\end{acknowledgments}

\bibliographystyle{gji}
\bibliography{tpw_rate.bib}


\appendix

\section{Degree-two moments}

We may explicitly draw a connection between the moment of inertia of a rotating object and it's degree-two density structure.  The moment of inertia tensor may be written in index notation
\begin{equation}
I_{ij} = \int_V \rho \left( r_q r_q \delta_{ij} - r_i r_j \right) d^3 {\bf r}
\label{inertia}
\end{equation}

where $\bf r$ is the Eulerian coordinate, $\rho$ is the density, and $V$ is the volume of the material.  The degree-two density structure may be expressed as a quadrupole-moment tensor (cf. \citet{jackson1998classical}):

\begin{equation}
Q_{ij} = \int_V \rho \left( 3 r_i r_j - r_q r_q \delta_{ij} \right) d^3 {\bf r}
\end{equation}

The commutator of these tensors, $\bf QI - IQ$, is zero, indicating that they may be simultaneously diagonalized.  

\ifdetail
We may demonstrate that the commutator is zero by using the definition of the commutator and expanding:
\begin{equation} 
\begin{aligned}
{\bf IQ-QI} = &I_{ik} Q_{kj} - Q_{ik} I_{kj} \\
 = &\int_V \rho \left( r_q r_q \delta_{ik} - r_i r_k \right) d^3 {\bf r} \int_V \rho \left( 3 r_k r_j - r_q r_q \delta_{kj} \right) d^3 {\bf r} \\
 &- \int_V \rho \left( r_q r_q \delta_{kj} - r_k r_j \right) d^3 {\bf r} \int_V \rho \left( 3 r_i r_k - r_q r_q \delta_{ik} \right) d^3 {\bf r} \\
 = &3 \delta_{ik} \int_V \rho r_q r_q d^3 {\bf r} \int_V \rho r_k r_j d^3 {\bf r} \\
 &-\delta_{kj} \delta_{ik} \int_V \rho r_q r_q d^3 {\bf r} \int_V \rho r_q r_q d^3 {\bf r} \\
 &-3 \int_V \rho r_i r_k d^3 {\bf r} \int_V \rho r_k r_j d^3 {\bf r} \\
 &+\delta_{kj} \int_V \rho r_i r_k d^3 {\bf r} \int_V \rho r_q r_q d^3 {\bf r} \\
 - &3 \delta_{kj} \int_V \rho r_q r_q d^3 {\bf r} \int_V \rho r_i r_k d^3 {\bf r} \\
 &+\delta_{ik} \delta_{kj} \int_V \rho r_q r_q d^3 {\bf r} \int_V \rho r_q r_q d^3 {\bf r} \\
 &+3 \int_V \rho r_k r_j d^3 {\bf r} \int_V \rho r_i r_k d^3 {\bf r} \\
 &-\delta_{ik} \int_V \rho r_k r_j d^3 {\bf r} \int_V \rho r_q r_q d^3 {\bf r} \\
 = \;&3 \int_V \rho r_q r_q d^3 {\bf r} \int_V \rho r_i r_j d^3 {\bf r} \\
 &+ \int_V \rho r_i r_j d^3 {\bf r} \int_V \rho r_q r_q d^3 {\bf r} \\
 - &3 \int_V \rho r_q r_q d^3 {\bf r} \int_V \rho r_i r_j d^3 {\bf r} \\
 &- \int_V \rho r_i r_j d^3 {\bf r} \int_V \rho r_q r_q d^3 {\bf r} \\
 & = 0
\end{aligned}
\end{equation}
\else
\fi

We may therefore go into the principle coordinate system of both tensors to find
\begin{equation}
{\bf I} = \begin{bmatrix}
\int_V \rho (y^2+z^z)\\
\int_V \rho (x^2+z^2) \\
\int_V \rho (x^2+y^2) 
\end{bmatrix}^T {\bf 1}
\end{equation}

\begin{equation}
{\bf Q} = \begin{bmatrix}
\int_V \rho (2 x^2 - y^2 - z^z)\\
\int_V \rho (2 y^2 - x^2 -z^2)\\
\int_V \rho (2 z^2 - x^2 - y^2) 
\end{bmatrix}^T {\bf 1}
\end{equation}

where $\bf 1$ is the identity matrix.  If we let the diagonal elements of $\bf I$ be $\lambda_1, \lambda_2, \lambda_3$, we may rewrite $\bf Q$ as

\begin{equation}
{\bf Q} = \begin{bmatrix}
(\lambda_2-\lambda_1) + (\lambda_3-\lambda_1)\\
(\lambda_1-\lambda_2) + (\lambda_3-\lambda_2)\\
(\lambda_1-\lambda_3) + (\lambda_2-\lambda_3)\\
\end{bmatrix}^T {\bf 1}
\label{maccullagh}
\end{equation}

which shows the connection between the degree-two component of the density field and the principle moments of inertia.  This may be seen as a form of MacCullagh's formula \citep{stacey1977physics}.

\label{lastpage}

\end{document}
