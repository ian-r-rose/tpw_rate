\documentclass[a4paper,12pt]{article}
\usepackage[utf8]{inputenc}

%opening
\title{Response to reviewer 1 (Victor Tsai)}
\author{}
\date{}

\begin{document}

\maketitle

We thank Victor for his comprehensive review of our paper.
We agree that there are new and intriguing results in this paper.
We also acknowledge that the original manuscript lacked clarity on several key concepts.
We also value and appreciate the suggestion of making greater points of contact with the current literature.
The revised manuscript has been substantially modified to address all of the reviewer's comments.
We have elaborated on key points, expanded our discussion of prior work, and altered the organization of the paper such that each new section builds directly on the earlier results.
All of these changes have improved the presentation.
We think these changes will also erase any confusion between $\Theta$ and $\theta$.


The confusion between $\theta$ and $\Theta$ is partly connected to the definition of inertial interchange true polar wander (IITPW).
Victor's view is clearly expressed in his review:
\begin{quote}
Only if Theta eventually ends up close to 90 degrees would it be IITPW.
\end{quote}
We have a slightly different view, but this difference is largely a question of semantics.
To us an inertial interchange occurs when the maximum and intermediate moments of inertia are interchanged.
Victor describes IITPW as the long-time consequence of the interchange.
Both perspectives are reasonable, and actually describe slightly different things, so the key to resolving this question in the paper is to emphasize the physics of the process.
On this point, we do not think there is any disagreement between us and the reviewer.
We have added text throughout the paper to address all of the points in the second paragraph of the review (lines 478, 483, 530) and earlier in the paper (e.g. line 104).
We also follow the advice of renaming one of the thetas.
We hesitate to annotate these changes in the text using a different colored font because these changes are fairly pervasive and (in places) extensive.

The third paragraph of the review deals with the difficulty for the reader to ``understand what the main results of the manuscript were''.
We readily acknowledge this point.
Several key concepts were developed in isolated, self-standing sections.
We impose a heavy burden on the readers by asking them to put the pieces together for themselves.
A reorganization of the paper permits a more systematic development of the ideas; each new results builds on the previous results, so the connections are much clearer.
We also place signposts at key points to ensure that the readers does not lose sight of the end goal.
The reviewer specifically mentions the appears of additional equations in the ``Discussion'' section.
We hope that these concerns are eliminated by the reorganization.

Victor goes on to make valuable suggestions in the fourth paragraph of the review about linking our work to existing literature.
This is a fair criticism.
This is not only reasonable, but it also helps to emphasize the main contributions.
We implement all of the specific suggestions connected with Tsai \& Stevenson (2007), and we make similar additions connected with the studies of 
Ricard et al. (1993); Richards et al. (1999) and Cambiotti et al. (2011).

Responses to the additional comments/suggestions:

\begin{enumerate}

\item We add commas to or invert the relevant sentences.

\item Caption to Fig. 1 is reworded to clarify.

\item Use of $T$ to denote temperature and $T_1$ denote relaxation time is avoiding by using $\tau$ for all timescales.

\item Line 396 does not assume $\theta$ to be small. We have elaborated on the cited literature to clarify this point. 

\item The reference to a theorem in D\&K1970 is move to the place where the theorem is stated.

\item The text around line 456 is substantially modified. It was possible to streamline the text and improve clarity.

\item Lines 468/469 were confusing. The text is revised and the typo is corrected.

\item How representative is Fig. 4? The answer is ``very''. There was no need to be selective about the examples we showed. The events in question are always present when convection is sufficiently vigorous. We completely agree that the rate of TPW is small when theta = 90. This fact is acknowledged in the revised manuscript and confirmed in our equation for the rate of TPW. However, embedding this equation in a mantle convection simulation reveals some interesting complications because the mass anomalies that cause TPW are always in motion.

\end{enumerate}

We appreciate Victor's thoughtful suggestions.

\end{document}
